\documentclass[man]{apa6}
\title{Aftermaths Misconceptions Regarding AI}
\author{Murat Ambarkutuk}
\affiliation{English Language Institute, University of Delaware \\ murata@udel.edu}
\note{In-Class Summary; 6/8/2015}
\usepackage{apacite}
\shorttitle{AI Misconceptions}
\leftheader{Even-Numbered Page Header} %Change it
\begin{document}
\maketitle
Since the rise of the machines, approximately coinciding with the Industrial Revolution, individuals and experts have tried position accurately machines into society. Some say, machines are just tools to automate simple tasks and mundane duties of humans, while others argue against that notion. In the article entitled "The First Church of Robotics", \citeA{lan} asserts that as technology advances, the by-products and the spin off technologies have started to be antromorphosized. \citeA{lan} sarcastically claims in his article title that Artificial Intelligence (AI)  and Robotics have been gradually becoming a "religion". Among many problems that the misconception of AI poses, promotion of loneliness, devaluation of information and distrust of human judgment will be the main foci of this essay.\par
Social interaction is what differentiates humans from machines. However, as the research studies toward creation of AI have gone deeper, robots have become more socially interactive in society. For instance, before GPS became publicly available, route planning was based upon map-reading and asking directions of the locals. In light of this example, AI has become route planning assistant and has decreased the inter-relation among individuals. According to \citeA{lan}, reformation of our perception towards AI poses misconceptions regarding personhood and promotes loneliness. This issue truly matters due to the fact that alienation from the society relies on that misconception. Depression and other similar psychological challenges are the most profound indicators of the aftermaths of inaccurate definition of AI.\par
Along with the alienation from society, another consequence of false personifcation of AI is the devaluation of information. Even though the Internet facilitates the availability of information in a cheaper and more efficient fashion, it also raises informational misguidance problems. Finding a definition to a term has become so easy that just typing a keyword into a search engine is enough. However, that "go and get it" approach can be considered somewhat unprofessional. For instance, typing "communism" into a search engine does not provide adequate information for the researcher. Likewise, an answer from a robot teacher to even a non-paradoxical question may be in the same level of proficiency. Thus, AI would neither be a competent teacher, nor a reliable source of information.\par
Coupled with the psychological and informational drawbacks of AI, economic problems are on the verge of occurring. Robots and AI supposedly represent the maxima of accuracy and repeatability which promotes reliance on AI. Robots and AI seem to outperform human cognitive and learning abilities in near future, albeit undermining the true value of failures. The fast and accurate reasoning sensibility that robots show challenges workers and laborers in industry. Perhaps this challenge may increase the profitability, but it also creates mistrust of human judgment. Furthermore, if this aggresive expansion continues, other occupations and professions will be under the same risk of mechanization. Not only are the job opportunities, but social elevation will also disappear. \par
\bibliographystyle{apacite}
\bibliography{bibliography}
\end{document}